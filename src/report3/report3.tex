\documentclass{scrartcl}
\usepackage{listings}
\usepackage{caption}
\usepackage{color}
\usepackage{booktabs}
\usepackage{lscape}
\usepackage{tabularx}

\newcounter{nalg}[section] % defines algorithm counter for section
\renewcommand{\thenalg}{\thesection .\arabic{nalg}} %defines appearance of the algorithm counter
\DeclareCaptionLabelFormat{algocaption}{Algorithm \thenalg} % defines a new caption label as Algorithm x.y

\lstnewenvironment{algorithm}[1][] %defines the algorithm listing environment
{   
    \refstepcounter{nalg} %increments algorithm number
    \captionsetup{labelformat=algocaption,labelsep=colon} %defines the
                                                          %caption
                                                          %setup for:
                                                          %it ises
                                                          %label
                                                          %format as
                                                          %the
                                                          %declared
                                                          %caption
                                                          %label above
                                                          %and makes
                                                          %label and
                                                          %caption
                                                          %text to be
                                                          %separated
                                                          %by a ':'
    \lstset{ %this is the stype
        frame=tB,
        numbers=left, 
        numberstyle=\tiny,
        basicstyle=\scriptsize, 
        keywordstyle=\color{black}\bfseries,
        escapeinside={(*}{*)},
        keywords={,input, output, return, datatype, function, in, if, else, foreach, while, begin, end, } 
        numbers=left,
        xleftmargin=.04\textwidth,
        #1 % this is to add specific settings to an usage of this environment (for instnce, the caption and referable label)
    }
}
{}





\author{Alexander Eisl (0250266), Peter Wiedermann (0025999)}

\date{\today}


\title{Heuristic Optimization Techniques \\ Exercise 3}

\begin{document}
\maketitle


\section{Local Search}
\label{sec:deterministic}
In this programming exercise, we build on our local search from the
previous exercise and implement a variable neighborhood search
heuristic.

%%% brauchen wir sowas?
\begin{algorithm}[caption={Variable Neighborhood Search}]
    input: Graph 
    output: Improved solution
    begin
        l (*$\leftarrow$*) 1;
	repeat:
    	find (*$x'$*) with (*$f(x') \leq f(x''), \forall x'' \in N_l(x) $*) calculate initial solution
        if (*f($x') < f(x)$*) then
           (*$x \leftarrow x'$*);
           l (*$\leftarrow$*) 1;
        else 
           l (*$\leftarrow$*) l + 1
	until  (*$l > l_{max}$*)
    end
\end{algorithm}


%%% %%%%%%%%%%%%%%%%%%%%%%%%%%%%%%%%%%%%%%%%%%%%%%%%%%%%%5
%%% NEIGHBORHOODS HERE
\subsection{Neighborhoods}
\label{neighborhoods}
We re-use the neighborhoods from assignment two and added a additional ones.

[TODO: should we use paramiterized ones?]

\begin{description}
\item[1-node flip] 
    
    This neighborhood is defined as all subsets where two vertices of the initial solution are flipped. 
    \begin{itemize}
        \item Size of neigborhood: $n(n-1)/2$
        \item Objective Function: Incremental, crossings from flipped vertices are subtracted and recalculated.

    \end{itemize}

	
\item[1-edge move] This neighborhood consists of all solutions where one edge is moved to a different page. 
     \begin{itemize}
        \item Size of neigborhood: $(pages-1) edges$ 
        \item Objective Function: Incremental, only crossings for moved edges are recalculated.
    \end{itemize}
   
	
\item[1-node edge move] This neighborhood consists of all solutions where the edges of
	one vertex on a specific page are moved to all different pages. 

    \begin{itemize}
        \item Size of neigborhood: $(pages-1) edges$ as worst case.\footnote{depending on number of edges on respective node pages}
        \item Objective Function: Incremental, only crossings for moved edges are recalculated.
    \end{itemize}
    

\end{description}

%%% %%%%%%%%%%%%%%%%%%%%%%%%%%%%%%%%%%%%%%%%%%%%%%%%%%%%%5
%%% STEPFUNCTIONS HERE
\subsection{Stepfunctions}

We also re-use the step-functions from assignment 2. For computational
reasons, we only use the \texttt{first} and \texttt{random}
step-functions for the large instances.

\begin{description}
\item[first] This step-function selects the first better solution found in the neighborhood.
\item[best] This step-function iterates over all elements of the neighborhood and chooses the best one.
\item[random] This step-function selects a random element of the neighborhood. 
\end{description}


%%% REMOVED, I don't think this is necessary
%%\section{Solution Representation}
%%The same as in exercise 1. Do we need this subsection?


%%% %%%%%%%%%%%%%%%%%%%%%%%%%%%%%%%%%%%%%%%%%%%%%%%%%%%%%5
%%% NEIGHBORHOODS HERE
\section{Results}

We executed the code on a desktop computer with a Core i7 Quad-Core
CPU with 2.67Ghz and 24 GB of main memory. Table~\ref{tab:results}
shows the results of our local search. 

We consider a local optimum if we cannot find a better solution after 10 iterations.
We have set a timout of 5 minutes for the calculations.

\paragraph{Best neighborhood step function strategy}


%Overall, the combination of the 1-node-flip neighborhood and the best
%step-function shows the best improvements. For example, it finds a
%solution with only $12$ edge-crossings for the instance automatic-4.
%Not surprisingly, it also has the highest run-time.




\paragraph{Order of the neighborhoods}

TODO: I would guess it does matter... we probably should run some experiments...









\begin{landscape}
\begin{table}
  \scriptsize
  % latex table generated in R 2.15.1 by xtable 1.5-6 package
% Sat Nov  7 17:18:10 2015
\begin{tabular}{lccccccccc}
  \toprule  & \begin{minipage}[c]{1.4cm}\centering 1-Node \\ (first)\end{minipage} & \begin{minipage}[c]{1.4cm}\centering 1-Node \\ (best)\end{minipage} & \begin{minipage}[c]{1.4cm}\centering 1-Node \\ (random)\end{minipage} & \begin{minipage}[c]{1.4cm}\centering 1-Edge \\ (first)\end{minipage} & \begin{minipage}[c]{1.4cm}\centering 1-Edge \\ (best)\end{minipage} & \begin{minipage}[c]{1.4cm}\centering 1-Edge \\ (random)\end{minipage} & \begin{minipage}[c]{1.4cm}\centering 1-N-E \\ (first)\end{minipage} & \begin{minipage}[c]{1.4cm}\centering 1-N-E \\ (best)\end{minipage} & \begin{minipage}[c]{1.4cm}\centering 1-N-E \\ (random)\end{minipage} \\ 
  \midrule automatic-1.txt & \vspace{0.02cm} \begin{minipage}[c]{1.5cm} \centering 21\\0 / 0.02 \end{minipage} & \vspace{0.02cm} \begin{minipage}[c]{1.5cm} \centering 21\\0 / 0.00 \end{minipage} & \vspace{0.02cm} \begin{minipage}[c]{1.5cm} \centering 21\\0 / 0.01 \end{minipage} & \vspace{0.02cm} \begin{minipage}[c]{1.5cm} \centering 21\\0 / 0.00 \end{minipage} & \vspace{0.02cm} \begin{minipage}[c]{1.5cm} \centering 21\\0 / 0.00 \end{minipage} & \vspace{0.02cm} \begin{minipage}[c]{1.5cm} \centering 21\\0 / 0.01 \end{minipage} & \vspace{0.02cm} \begin{minipage}[c]{1.5cm} \centering 21\\0 / 0.00 \end{minipage} & \vspace{0.02cm} \begin{minipage}[c]{1.5cm} \centering 21\\0 / 0.00 \end{minipage} & \vspace{0.02cm} \begin{minipage}[c]{1.5cm} \centering 21\\0 / 0.00 \end{minipage} \\ 
    \midrule automatic-2.txt & \vspace{0.02cm} \begin{minipage}[c]{1.5cm} \centering 30\\4 / 0.07 \end{minipage} & \vspace{0.02cm} \begin{minipage}[c]{1.5cm} \centering 48\\0 / 0.00 \end{minipage} & \vspace{0.02cm} \begin{minipage}[c]{1.5cm} \centering 48\\0 / 0.00 \end{minipage} & \vspace{0.02cm} \begin{minipage}[c]{1.5cm} \centering 48\\0 / 0.00 \end{minipage} & \vspace{0.02cm} \begin{minipage}[c]{1.5cm} \centering 48\\0 / 0.00 \end{minipage} & \vspace{0.02cm} \begin{minipage}[c]{1.5cm} \centering 48\\0 / 0.00 \end{minipage} & \vspace{0.02cm} \begin{minipage}[c]{1.5cm} \centering 48\\0 / 0.00 \end{minipage} & \vspace{0.02cm} \begin{minipage}[c]{1.5cm} \centering 48\\0 / 0.00 \end{minipage} & \vspace{0.02cm} \begin{minipage}[c]{1.5cm} \centering 48\\0 / 0.01 \end{minipage} \\ 
    \midrule automatic-3.txt & \vspace{0.02cm} \begin{minipage}[c]{1.5cm} \centering 101\\6 / 0.30 \end{minipage} & \vspace{0.02cm} \begin{minipage}[c]{1.5cm} \centering 131\\0 / 0.00 \end{minipage} & \vspace{0.02cm} \begin{minipage}[c]{1.5cm} \centering 119\\3 / 0.05 \end{minipage} & \vspace{0.02cm} \begin{minipage}[c]{1.5cm} \centering 115\\1 / 0.02 \end{minipage} & \vspace{0.02cm} \begin{minipage}[c]{1.5cm} \centering 131\\0 / 0.00 \end{minipage} & \vspace{0.02cm} \begin{minipage}[c]{1.5cm} \centering 131\\0 / 0.00 \end{minipage} & \vspace{0.02cm} \begin{minipage}[c]{1.5cm} \centering 131\\0 / 0.00 \end{minipage} & \vspace{0.02cm} \begin{minipage}[c]{1.5cm} \centering 131\\0 / 0.00 \end{minipage} & \vspace{0.02cm} \begin{minipage}[c]{1.5cm} \centering 80\\16 / 0.14 \end{minipage} \\ 
    \midrule automatic-4.txt & \vspace{0.02cm} \begin{minipage}[c]{1.5cm} \centering 7\\22 / 0.81 \end{minipage} & \vspace{0.02cm} \begin{minipage}[c]{1.5cm} \centering 143\\3 / 0.00 \end{minipage} & \vspace{0.02cm} \begin{minipage}[c]{1.5cm} \centering 141\\2 / 0.01 \end{minipage} & \vspace{0.02cm} \begin{minipage}[c]{1.5cm} \centering 124\\6 / 0.03 \end{minipage} & \vspace{0.02cm} \begin{minipage}[c]{1.5cm} \centering 165\\0 / 0.00 \end{minipage} & \vspace{0.02cm} \begin{minipage}[c]{1.5cm} \centering 128\\8 / 0.01 \end{minipage} & \vspace{0.02cm} \begin{minipage}[c]{1.5cm} \centering 169\\0 / 0.00 \end{minipage} & \vspace{0.02cm} \begin{minipage}[c]{1.5cm} \centering 169\\0 / 0.00 \end{minipage} & \vspace{0.02cm} \begin{minipage}[c]{1.5cm} \centering 128\\8 / 0.02 \end{minipage} \\ 
    \midrule automatic-5.txt & \vspace{0.02cm} \begin{minipage}[c]{1.5cm} \centering 36\\7 / 0.35 \end{minipage} & \vspace{0.02cm} \begin{minipage}[c]{1.5cm} \centering 74\\0 / 0.00 \end{minipage} & \vspace{0.02cm} \begin{minipage}[c]{1.5cm} \centering 67\\2 / 0.02 \end{minipage} & \vspace{0.02cm} \begin{minipage}[c]{1.5cm} \centering 72\\0 / 0.00 \end{minipage} & \vspace{0.02cm} \begin{minipage}[c]{1.5cm} \centering 73\\0 / 0.00 \end{minipage} & \vspace{0.02cm} \begin{minipage}[c]{1.5cm} \centering 74\\0 / 0.00 \end{minipage} & \vspace{0.02cm} \begin{minipage}[c]{1.5cm} \centering 74\\0 / 0.00 \end{minipage} & \vspace{0.02cm} \begin{minipage}[c]{1.5cm} \centering 74\\0 / 0.00 \end{minipage} & \vspace{0.02cm} \begin{minipage}[c]{1.5cm} \centering 51\\13 / 0.04 \end{minipage} \\ 
    \midrule automatic-6.txt & \vspace{0.02cm} \begin{minipage}[c]{1.5cm} \centering 6,151,784\\-1 / 316.34 \end{minipage} & \vspace{0.02cm} \begin{minipage}[c]{1.5cm} \centering 6,153,059\\0 / 15.69 \end{minipage} & \vspace{0.02cm} \begin{minipage}[c]{1.5cm} \centering 6,025,236\\-1 / 316.52 \end{minipage} & \vspace{0.02cm} \begin{minipage}[c]{1.5cm} \centering 6,153,059\\0 / 257.34 \end{minipage} & \vspace{0.02cm} \begin{minipage}[c]{1.5cm} \centering 6,153,059\\0 / 20.68 \end{minipage} & \vspace{0.02cm} \begin{minipage}[c]{1.5cm} \centering 6,036,408\\-1 / 316.29 \end{minipage} & \vspace{0.02cm} \begin{minipage}[c]{1.5cm} \centering 6,153,059\\0 / 16.52 \end{minipage} & \vspace{0.02cm} \begin{minipage}[c]{1.5cm} \centering 6,152,675\\-1 / 338.98 \end{minipage} & \vspace{0.02cm} \begin{minipage}[c]{1.5cm} \centering 6,151,720\\-1 / 317.04 \end{minipage} \\ 
    \midrule automatic-7.txt & \vspace{0.02cm} \begin{minipage}[c]{1.5cm} \centering 138,729\\-1 / 301.17 \end{minipage} & \vspace{0.02cm} \begin{minipage}[c]{1.5cm} \centering 141,314\\0 / 0.20 \end{minipage} & \vspace{0.02cm} \begin{minipage}[c]{1.5cm} \centering 140,262\\16 / 20.81 \end{minipage} & \vspace{0.02cm} \begin{minipage}[c]{1.5cm} \centering 141,307\\0 / 0.20 \end{minipage} & \vspace{0.02cm} \begin{minipage}[c]{1.5cm} \centering 141,314\\0 / 0.67 \end{minipage} & \vspace{0.02cm} \begin{minipage}[c]{1.5cm} \centering 139,435\\19 / 6.50 \end{minipage} & \vspace{0.02cm} \begin{minipage}[c]{1.5cm} \centering 141,314\\0 / 0.19 \end{minipage} & \vspace{0.02cm} \begin{minipage}[c]{1.5cm} \centering 141,314\\0 / 0.81 \end{minipage} & \vspace{0.02cm} \begin{minipage}[c]{1.5cm} \centering 137,924\\127 / 118.47 \end{minipage} \\ 
    \midrule automatic-8.txt & \vspace{0.02cm} \begin{minipage}[c]{1.5cm} \centering 571,516\\-1 / 304.24 \end{minipage} & \vspace{0.02cm} \begin{minipage}[c]{1.5cm} \centering 573,754\\0 / 3.34 \end{minipage} & \vspace{0.02cm} \begin{minipage}[c]{1.5cm} \centering 537,857\\-1 / 304.27 \end{minipage} & \vspace{0.02cm} \begin{minipage}[c]{1.5cm} \centering 573,668\\3 / 44.36 \end{minipage} & \vspace{0.02cm} \begin{minipage}[c]{1.5cm} \centering 573,754\\0 / 3.66 \end{minipage} & \vspace{0.02cm} \begin{minipage}[c]{1.5cm} \centering 552,019\\-1 / 304.22 \end{minipage} & \vspace{0.02cm} \begin{minipage}[c]{1.5cm} \centering 573,754\\0 / 3.30 \end{minipage} & \vspace{0.02cm} \begin{minipage}[c]{1.5cm} \centering 573,754\\0 / 42.91 \end{minipage} & \vspace{0.02cm} \begin{minipage}[c]{1.5cm} \centering 571,732\\-1 / 304.25 \end{minipage} \\ 
    \midrule automatic-9.txt & \vspace{0.02cm} \begin{minipage}[c]{1.5cm} \centering 1,252,209\\-1 / 304.73 \end{minipage} & \vspace{0.02cm} \begin{minipage}[c]{1.5cm} \centering 1,273,121\\0 / 3.79 \end{minipage} & \vspace{0.02cm} \begin{minipage}[c]{1.5cm} \centering 1,211,265\\-1 / 304.73 \end{minipage} & \vspace{0.02cm} \begin{minipage}[c]{1.5cm} \centering 1,273,071\\8 / 64.68 \end{minipage} & \vspace{0.02cm} \begin{minipage}[c]{1.5cm} \centering 1,273,121\\0 / 10.34 \end{minipage} & \vspace{0.02cm} \begin{minipage}[c]{1.5cm} \centering 1,244,554\\-1 / 304.66 \end{minipage} & \vspace{0.02cm} \begin{minipage}[c]{1.5cm} \centering 1,273,121\\0 / 3.72 \end{minipage} & \vspace{0.02cm} \begin{minipage}[c]{1.5cm} \centering 1,272,439\\3 / 58.45 \end{minipage} & \vspace{0.02cm} \begin{minipage}[c]{1.5cm} \centering 1,271,380\\-1 / 304.70 \end{minipage} \\ 
    \midrule automatic-10.txt & \vspace{0.02cm} \begin{minipage}[c]{1.5cm} \centering 56,831\\-1 / 302.44 \end{minipage} & \vspace{0.02cm} \begin{minipage}[c]{1.5cm} \centering 56,936\\0 / 1.45 \end{minipage} & \vspace{0.02cm} \begin{minipage}[c]{1.5cm} \centering 56,131\\-1 / 302.42 \end{minipage} & \vspace{0.02cm} \begin{minipage}[c]{1.5cm} \centering 56,775\\31 / 33.81 \end{minipage} & \vspace{0.02cm} \begin{minipage}[c]{1.5cm} \centering 56,936\\0 / 1.64 \end{minipage} & \vspace{0.02cm} \begin{minipage}[c]{1.5cm} \centering 56,877\\1 / 1.46 \end{minipage} & \vspace{0.02cm} \begin{minipage}[c]{1.5cm} \centering 56,936\\0 / 1.47 \end{minipage} & \vspace{0.02cm} \begin{minipage}[c]{1.5cm} \centering 56,936\\0 / 6.69 \end{minipage} & \vspace{0.02cm} \begin{minipage}[c]{1.5cm} \centering 55,844\\-1 / 302.39 \end{minipage} \\ 
   \bottomrule \end{tabular}

\caption{Local search results. For each instance, we show the number
  of crossings (first row), the iteration needed to reach the local
  optimum (second row, first value) and the run-time in seconds of our
  algorithm (second row, second value).  If no local optimum was found
  the number of iterations is -1. The best solution is highlighted in
  bold.}
\label{tab:results}
\end{table}
\end{landscape}



\end{document}
