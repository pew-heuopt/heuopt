\documentclass{scrartcl}
\usepackage{listings}
\usepackage{caption}
\usepackage{color}
\usepackage{booktabs}

\newcounter{nalg}[section] % defines algorithm counter for section
\renewcommand{\thenalg}{\thesection .\arabic{nalg}} %defines appearance of the algorithm counter
\DeclareCaptionLabelFormat{algocaption}{Algorithm \thenalg} % defines a new caption label as Algorithm x.y

\lstnewenvironment{algorithm}[1][] %defines the algorithm listing environment
{   
    \refstepcounter{nalg} %increments algorithm number
    \captionsetup{labelformat=algocaption,labelsep=colon} %defines the
                                                          %caption
                                                          %setup for:
                                                          %it ises
                                                          %label
                                                          %format as
                                                          %the
                                                          %declared
                                                          %caption
                                                          %label above
                                                          %and makes
                                                          %label and
                                                          %caption
                                                          %text to be
                                                          %separated
                                                          %by a ':'
    \lstset{ %this is the stype
        frame=tB,
        numbers=left, 
        numberstyle=\tiny,
        basicstyle=\scriptsize, 
        keywordstyle=\color{black}\bfseries,
        escapeinside={(*}{*)},
        keywords={,input, output, return, datatype, function, in, if, else, foreach, while, begin, end, } 
        numbers=left,
        xleftmargin=.04\textwidth,
        #1 % this is to add specific settings to an usage of this environment (for instnce, the caption and referable label)
    }
}
{}





\author{Alexander Eisl (0250266), Peter Wiedermann (0025999)}

\date{\today}


\title{Heuristic Optimization Techniques \\ Exercise 2}

\begin{document}
\maketitle


\section{Local Search}
\label{sec:deterministic}
In this programming exercise, we improve the results of the construction
heuristic developed in the previous exercise by using a local search. In the following
subsection we will first explain the neighborhoods that we define and then outline the
step functions used.

%%% brauchen wir sowas?
\begin{algorithm}[caption={Local search}]
    input: Initial solution
    output: Improved solution
    begin
    	sol = initial solution;
	repeat:
		choose an sol' $\in$ N(x) 
		if f($sol'$) $\leq$ f(sol) then
			sol = $sol'$
	until stopping criteria satisfied
    end
\end{algorithm}

\subsection{Neighborhoods}
\label{neighborhoods}
We use the following neighbourhoods:

\begin{description}
\item[1-node flip] 
    
    This neighbourhood is defined as all subsets where two vertices of the initial solution are flipped. 
    \begin{itemize}
        \item Size of neigborhood: $n(n-1)/2$
        \item Objective Function: Very intensive becuase we change the spine order.

    \end{itemize}

    Combination of big neigborhood size and complex objective function makes this  a omputationally intensive variant.
    However, we expect this to give large improvements at least for small graphs.

    TODO: number of steps to reach local optimum
	
\item[1-edge move] This neighbourhood consists of all solutions where one edge is moved to a different page. 
     \begin{itemize}
        \item Size of neigborhood: $(pages-1) edges$ 
        \item Objective Function: Incremental, only crossings for moved edges are recalculated.
    \end{itemize}
   
    The incremental objective function and the moderat neighborhood size makes this variant computational feasible.

    TODO: number of steps to reach local optimum
	
\item[1-node edge move] This neighbourhood consists of all solutions where the edges of
	one vertex on a specific page are moved to all different pages. 

    \begin{itemize}
        \item Size of neigborhood: $(pages-1) edges$ as worst case \footnote{depending on number of edges on respective node pages}.
        \item Objective Function: Incremental, only crossings for moved edges are recalculated.
    \end{itemize}
    
    This also allows for incremental neighbourhoods, the
		size is $(pages-1) |N(v_i)|$

\end{description}


\subsection{Stepfunctions}

We implemented the following thre stepfunctions:

\begin{description}
\item[first] 
\item[best] 
\item[random] 
\end{description}



\section{Solution Representation}
The same as in exercise 1. Do we need this subsection?

\section{Results}
%%% den doppelten Verweis auf Table 3 gestrichen
We executed the code on a desktop computer with a Core i7 Quad-Core
CPU with 2.67Ghz and 24 GB of main memory. Table~\ref{tab:results}
shows the results of our randomized algorithm, where we have choosen
to perform $n=4$ permutations of vertices to generete $k=5$ different
solutions.


improve our deterministic approach.

Q1: Does using a good initial solution help? \\
A1: It's very important to have a good initial solution. First, because it helps to end up
in a good local optimum. Second, because it significantly reduces the run-time of the algorithm.

TODO: wo steht Q1?

Does it yield a gain if compared to simply using random initial solutions?
TODO: Use 1-node-flip and best stepfunction to answer this question because it should show the best effect


Q2: Can subsequent (possibly random) moves in your neighbourhood(s) reach
any valid solution in the search space?
A2: ?

Q3: How many iterations does it take to reach local optima? What does this say about your neighbourhood(s)?

Q4: How does incremental evaluation work for your neighbourhoods?
see description of neigborhoods in section \ref{neighborhoods}.




\end{document}
