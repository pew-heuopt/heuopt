\documentclass{scrartcl}
\usepackage{listings}
\usepackage{caption}
\usepackage{color}
\usepackage{booktabs}
\usepackage{lscape}
\usepackage{tabularx}

\newcounter{nalg}[section] % defines algorithm counter for section
\renewcommand{\thenalg}{\thesection .\arabic{nalg}} %defines appearance of the algorithm counter
\DeclareCaptionLabelFormat{algocaption}{Algorithm \thenalg} % defines a new caption label as Algorithm x.y

\lstnewenvironment{algorithm}[1][] %defines the algorithm listing environment
{   
    \refstepcounter{nalg} %increments algorithm number
    \captionsetup{labelformat=algocaption,labelsep=colon} %defines the
                                                          %caption
                                                          %setup for:
                                                          %it ises
                                                          %label
                                                          %format as
                                                          %the
                                                          %declared
                                                          %caption
                                                          %label above
                                                          %and makes
                                                          %label and
                                                          %caption
                                                          %text to be
                                                          %separated
                                                          %by a ':'
    \lstset{ %this is the stype
        frame=tB,
        numbers=left, 
        numberstyle=\tiny,
        basicstyle=\scriptsize, 
        keywordstyle=\color{black}\bfseries,
        escapeinside={(*}{*)},
        keywords={,input, output, return, datatype, function, in, if, else, foreach, while, begin, end, } 
        numbers=left,
        xleftmargin=.04\textwidth,
        #1 % this is to add specific settings to an usage of this environment (for instnce, the caption and referable label)
    }
}
{}





\author{Alexander Eisl (0250266), Peter Wiedermann (0025999)}

\date{\today}


\title{Heuristic Optimization Techniques \\ Exercise 2}

\begin{document}
\maketitle


\section{Local Search}
\label{sec:deterministic}
In this programming exercise, we improve the results of the
construction heuristic developed in the previous exercise by using a
local search. In the following subsection, we will first explain the
neighborhoods that we define and then outline the step functions used.

%%% brauchen wir sowas?
\begin{algorithm}[caption={Local search}]
    input: Graph 
    output: Improved solution
    begin
    	sol (*$\leftarrow$*) calculate initial solution
	repeat:
		choose sol' (*$\in$*) N(x) 
		if (*$f(sol')$*) (*$\leq$*) (*$f(sol)$*) then
			sol (*$\leftarrow$*) sol'
	until stopping criteria satisfied
    end
\end{algorithm}


%%% %%%%%%%%%%%%%%%%%%%%%%%%%%%%%%%%%%%%%%%%%%%%%%%%%%%%%5
%%% NEIGHBORHOODS HERE
\subsection{Neighborhoods}
\label{neighborhoods}
We use the following neighborhoods:

\begin{description}
\item[1-node flip] 
    
    This neighborhood is defined as all subsets where two vertices of the initial solution are flipped. 
    \begin{itemize}
        \item Size of neigborhood: $n(n-1)/2$
        \item Objective Function: Incremental, crossings from flipped vertices are subtracted and recalculated.

    \end{itemize}

	
\item[1-edge move] This neighborhood consists of all solutions where one edge is moved to a different page. 
     \begin{itemize}
        \item Size of neigborhood: $(pages-1) edges$ 
        \item Objective Function: Incremental, only crossings for moved edges are recalculated.
    \end{itemize}
   
	
\item[1-node edge move] This neighborhood consists of all solutions where the edges of
	one vertex on a specific page are moved to all different pages. 

    \begin{itemize}
        \item Size of neigborhood: $(pages-1) edges$ as worst case.\footnote{depending on number of edges on respective node pages}
        \item Objective Function: Incremental, only crossings for moved edges are recalculated.
    \end{itemize}
    

\end{description}

%%% %%%%%%%%%%%%%%%%%%%%%%%%%%%%%%%%%%%%%%%%%%%%%%%%%%%%%5
%%% STEPFUNCTIONS HERE
\subsection{Stepfunctions}

We implemented the following thre stepfunctions:

\begin{description}
\item[first] This step-function selects the first better solution found in the neighborhood.
\item[best] This step-function iterates over all elements of the neighborhood and chooses the best one.
\item[random] This step-function selects a random element of the neighborhood. 
\end{description}


%%% REMOVED, I don't think this is necessary
%%\section{Solution Representation}
%%The same as in exercise 1. Do we need this subsection?


%%% %%%%%%%%%%%%%%%%%%%%%%%%%%%%%%%%%%%%%%%%%%%%%%%%%%%%%5
%%% NEIGHBORHOODS HERE
\section{Results}

We executed the code on a desktop computer with a Core i7 Quad-Core
CPU with 2.67Ghz and 24 GB of main memory. Table~\ref{tab:results}
shows the results of our local search. 

We consider a local optimum if we cannot find a better solution after 10 iterations.
We have set a timout of 5 minutes for the calculations.

\paragraph{Best neighborhood step function strategy}

We compare the results for three different neighboorhoods and all the
implemented step-functions.

Depending on the size of the instance, either \texttt{1-node-flip} or
\texttt{1-node-edge-move} yield the best results. These neighborhoods
outperfom \texttt{1-edge-move}, which only moves edges but leaves the
vertex order unchanged, in most cases. For smaller instances (1-5), the
\texttt{1-node-flip} in combination with the \texttt{best} step-function delivers
the best results.  For bigger instances (6-10), the
\texttt{1-node-flip} finds better solutions in combination with the
\texttt{random} step-function.\footnote{Note that for large instances,
  \texttt{1-node-flip} and \texttt{1-node-edge move} are stopped by
  our run-time restrictions. Thus, these strategies might be too
  computationally intensive for our setup.}

A very interesting result is the performance of the \texttt{best}
step-function for both the \texttt{1-node-flip} and the
\texttt{1-node-edge move}. For the given instances, it seems to be
important to look for the best improvement and the first step as
\texttt{first} and \texttt{random} tend to get stuck in worse local
optima.

%Overall, the combination of the 1-node-flip neighborhood and the best
%step-function shows the best improvements. For example, it finds a
%solution with only $12$ edge-crossings for the instance automatic-4.
%Not surprisingly, it also has the highest run-time.




\paragraph{Initial Solution}
The choice of the initial solution is crucial to achieve results in
reasonable calculation time.  Using random initialization might
degenerate into random search or end in a bad local optimum with less
improvement.



% TODO:
When we compare the results to a random initial solution, we could
measure an average increase of 55\% crossings.\footnote{measured with 1-node flip and \texttt{first} stepfunction}
%

%instance    lsr-1node-first-rand    lsr-1node-first-init    sub         
%automatic-1.txt 21  21  0           
%automatic-2.txt 83  48  35  0.7291666667        
%automatic-3.txt 147 115 32  0.2782608696        
%automatic-4.txt 119 124 5   0.0403225806        
%automatic-5.txt 109 72  37  0.5138888889        
%automatic-6.txt 7524000 6153059 1370941 0.2228064122        
%automatic-7.txt 162956  141307  21649   0.1532054321        
%automatic-8.txt 1046196 573668  472528  0.8236959356        
%automatic-9.txt 1547842 1273071 274771  0.2158332096        
%automatic-10.txt    201019  56775   144244  2.5406252752        
%                                            5.5178052706    10  0.5517805271
%

% OLD:

%It is important to use a good initial solution
%. First, because this
%helps to end up in a good local solution.\footnote{For example, the
%  results shown on the AC cluster indicate that it should be possible
%  to find a solution with zero crossings. However, our local solution
%  is not able to find them given the initial solutions provided by our
%  construction heuristic.} Second, it can reduce the run-time of the
%algorithm because it might reduce the number of steps taken to reach
%the local optimum.


\paragraph{Neighborhoods}
Incremental objective functions implemented as described in Section~\ref{neighborhoods}
are performing equally well for all three neighborhoods.

Subsequent (possibly random) moves in our neighborhoods can only reach
valid solutions in the search space because every solution in our neighborhoods
is valid.

\paragraph{Reaching local optimum}
If a local optimum is reached it is found in less than 10 iterations
in most cases.  As can be seen in Table~\ref{tab:results}, there are
only a few exceptions. The results with iterations=$-1$ did not reach
a local optimum and ran into our time-out condition. For one instance,
we do not find an improvement. So with respect to our neighborhood, we
seem to be in a local optimum.







\begin{landscape}
\begin{table}
  \scriptsize
  % latex table generated in R 3.0.2 by xtable 1.7-4 package
% Sat Nov  7 12:03:13 2015
\begin{tabular}{lccccccccc}
  \toprule  & \begin{minipage}[c]{1.4cm}\centering 1-Node \\ (first)\end{minipage} & \begin{minipage}[c]{1.4cm}\centering 1-Node \\ (best)\end{minipage} & \begin{minipage}[c]{1.4cm}\centering 1-Node \\ (random)\end{minipage} & \begin{minipage}[c]{1.4cm}\centering 1-Edge \\ (first)\end{minipage} & \begin{minipage}[c]{1.4cm}\centering 1-Edge \\ (best)\end{minipage} & \begin{minipage}[c]{1.4cm}\centering 1-Edge \\ (random)\end{minipage} & \begin{minipage}[c]{1.4cm}\centering 1-N-E \\ (first)\end{minipage} & \begin{minipage}[c]{1.4cm}\centering 1-N-E \\ (best)\end{minipage} & \begin{minipage}[c]{1.4cm}\centering 1-N-E \\ (random)\end{minipage} \\ 
  \midrule automatic-1.txt & \vspace{0.02cm} \begin{minipage}[c]{1.5cm} \centering 21\\0 / 0.00 \end{minipage} & \vspace{0.02cm} \begin{minipage}[c]{1.5cm} \centering 21\\0 / 0.00 \end{minipage} & \vspace{0.02cm} \begin{minipage}[c]{1.5cm} \centering 21\\0 / 0.00 \end{minipage} & \vspace{0.02cm} \begin{minipage}[c]{1.5cm} \centering 21\\0 / 0.00 \end{minipage} & \vspace{0.02cm} \begin{minipage}[c]{1.5cm} \centering 21\\0 / 0.00 \end{minipage} & \vspace{0.02cm} \begin{minipage}[c]{1.5cm} \centering 21\\0 / 0.00 \end{minipage} & \vspace{0.02cm} \begin{minipage}[c]{1.5cm} \centering 21\\0 / 0.00 \end{minipage} & \vspace{0.02cm} \begin{minipage}[c]{1.5cm} \centering 21\\0 / 0.00 \end{minipage} & \vspace{0.02cm} \begin{minipage}[c]{1.5cm} \centering 21\\0 / 0.00 \end{minipage} \\ 
    \midrule automatic-2.txt & \vspace{0.02cm} \begin{minipage}[c]{1.5cm} \centering 48\\0 / 0.00 \end{minipage} & \vspace{0.02cm} \begin{minipage}[c]{1.5cm} \centering 30\\4 / 0.04 \end{minipage} & \vspace{0.02cm} \begin{minipage}[c]{1.5cm} \centering 47\\0 / 0.00 \end{minipage} & \vspace{0.02cm} \begin{minipage}[c]{1.5cm} \centering 48\\0 / 0.00 \end{minipage} & \vspace{0.02cm} \begin{minipage}[c]{1.5cm} \centering 48\\0 / 0.00 \end{minipage} & \vspace{0.02cm} \begin{minipage}[c]{1.5cm} \centering 48\\0 / 0.00 \end{minipage} & \vspace{0.02cm} \begin{minipage}[c]{1.5cm} \centering 48\\0 / 0.00 \end{minipage} & \vspace{0.02cm} \begin{minipage}[c]{1.5cm} \centering 48\\0 / 0.00 \end{minipage} & \vspace{0.02cm} \begin{minipage}[c]{1.5cm} \centering 48\\0 / 0.00 \end{minipage} \\ 
    \midrule automatic-3.txt & \vspace{0.02cm} \begin{minipage}[c]{1.5cm} \centering 104\\0 / 0.01 \end{minipage} & \vspace{0.02cm} \begin{minipage}[c]{1.5cm} \centering 94\\5 / 0.21 \end{minipage} & \vspace{0.02cm} \begin{minipage}[c]{1.5cm} \centering 104\\0 / 0.00 \end{minipage} & \vspace{0.02cm} \begin{minipage}[c]{1.5cm} \centering 104\\0 / 0.00 \end{minipage} & \vspace{0.02cm} \begin{minipage}[c]{1.5cm} \centering 79\\8 / 0.07 \end{minipage} & \vspace{0.02cm} \begin{minipage}[c]{1.5cm} \centering 104\\0 / 0.01 \end{minipage} & \vspace{0.02cm} \begin{minipage}[c]{1.5cm} \centering 104\\0 / 0.00 \end{minipage} & \vspace{0.02cm} \begin{minipage}[c]{1.5cm} \centering 97\\2 / 0.04 \end{minipage} & \vspace{0.02cm} \begin{minipage}[c]{1.5cm} \centering 104\\0 / 0.00 \end{minipage} \\ 
    \midrule automatic-4.txt & \vspace{0.02cm} \begin{minipage}[c]{1.5cm} \centering 69\\36 / 0.28 \end{minipage} & \vspace{0.02cm} \begin{minipage}[c]{1.5cm} \centering 12\\17 / 0.66 \end{minipage} & \vspace{0.02cm} \begin{minipage}[c]{1.5cm} \centering 192\\0 / 0.00 \end{minipage} & \vspace{0.02cm} \begin{minipage}[c]{1.5cm} \centering 163\\6 / 0.00 \end{minipage} & \vspace{0.02cm} \begin{minipage}[c]{1.5cm} \centering 155\\9 / 0.02 \end{minipage} & \vspace{0.02cm} \begin{minipage}[c]{1.5cm} \centering 192\\0 / 0.00 \end{minipage} & \vspace{0.02cm} \begin{minipage}[c]{1.5cm} \centering 164\\9 / 0.01 \end{minipage} & \vspace{0.02cm} \begin{minipage}[c]{1.5cm} \centering 169\\4 / 0.01 \end{minipage} & \vspace{0.02cm} \begin{minipage}[c]{1.5cm} \centering 192\\0 / 0.00 \end{minipage} \\ 
    \midrule automatic-5.txt & \vspace{0.02cm} \begin{minipage}[c]{1.5cm} \centering 75\\0 / 0.00 \end{minipage} & \vspace{0.02cm} \begin{minipage}[c]{1.5cm} \centering 32\\10 / 0.44 \end{minipage} & \vspace{0.02cm} \begin{minipage}[c]{1.5cm} \centering 77\\0 / 0.00 \end{minipage} & \vspace{0.02cm} \begin{minipage}[c]{1.5cm} \centering 77\\0 / 0.00 \end{minipage} & \vspace{0.02cm} \begin{minipage}[c]{1.5cm} \centering 41\\15 / 0.04 \end{minipage} & \vspace{0.02cm} \begin{minipage}[c]{1.5cm} \centering 77\\0 / 0.00 \end{minipage} & \vspace{0.02cm} \begin{minipage}[c]{1.5cm} \centering 77\\0 / 0.00 \end{minipage} & \vspace{0.02cm} \begin{minipage}[c]{1.5cm} \centering 69\\1 / 0.01 \end{minipage} & \vspace{0.02cm} \begin{minipage}[c]{1.5cm} \centering 77\\0 / 0.00 \end{minipage} \\ 
    \midrule automatic-6.txt & \vspace{0.02cm} \begin{minipage}[c]{1.5cm} \centering 6,153,059\\0 / 128.80 \end{minipage} & \vspace{0.02cm} \begin{minipage}[c]{1.5cm} \centering 6,153,059\\0 / 129.39 \end{minipage} & \vspace{0.02cm} \begin{minipage}[c]{1.5cm} \centering 6,153,059\\0 / 14.15 \end{minipage} & \vspace{0.02cm} \begin{minipage}[c]{1.5cm} \centering 6,046,761\\0 / 128.70 \end{minipage} & \vspace{0.02cm} \begin{minipage}[c]{1.5cm} \centering 6,151,720\\0 / 128.59 \end{minipage} & \vspace{0.02cm} \begin{minipage}[c]{1.5cm} \centering 6,153,058\\0 / 166.99 \end{minipage} & \vspace{0.02cm} \begin{minipage}[c]{1.5cm} \centering 6,153,059\\0 / 7.74 \end{minipage} & \vspace{0.02cm} \begin{minipage}[c]{1.5cm} \centering 6,057,073\\0 / 128.63 \end{minipage} & \vspace{0.02cm} \begin{minipage}[c]{1.5cm} \centering 6,153,059\\0 / 8.30 \end{minipage} \\ 
    \midrule automatic-7.txt & \vspace{0.02cm} \begin{minipage}[c]{1.5cm} \centering 142,777\\32 / 91.50 \end{minipage} & \vspace{0.02cm} \begin{minipage}[c]{1.5cm} \centering 143,070\\0 / 121.16 \end{minipage} & \vspace{0.02cm} \begin{minipage}[c]{1.5cm} \centering 143,905\\0 / 0.34 \end{minipage} & \vspace{0.02cm} \begin{minipage}[c]{1.5cm} \centering 140,027\\60 / 12.58 \end{minipage} & \vspace{0.02cm} \begin{minipage}[c]{1.5cm} \centering 138,413\\0 / 121.14 \end{minipage} & \vspace{0.02cm} \begin{minipage}[c]{1.5cm} \centering 143,905\\0 / 1.16 \end{minipage} & \vspace{0.02cm} \begin{minipage}[c]{1.5cm} \centering 143,905\\0 / 0.17 \end{minipage} & \vspace{0.02cm} \begin{minipage}[c]{1.5cm} \centering 142,786\\19 / 15.77 \end{minipage} & \vspace{0.02cm} \begin{minipage}[c]{1.5cm} \centering 143,905\\0 / 0.16 \end{minipage} \\ 
    \midrule automatic-8.txt & \vspace{0.02cm} \begin{minipage}[c]{1.5cm} \centering 541,935\\0 / 122.75 \end{minipage} & \vspace{0.02cm} \begin{minipage}[c]{1.5cm} \centering 541,731\\0 / 122.83 \end{minipage} & \vspace{0.02cm} \begin{minipage}[c]{1.5cm} \centering 541,969\\0 / 2.75 \end{minipage} & \vspace{0.02cm} \begin{minipage}[c]{1.5cm} \centering 522,722\\0 / 122.75 \end{minipage} & \vspace{0.02cm} \begin{minipage}[c]{1.5cm} \centering 540,613\\0 / 122.71 \end{minipage} & \vspace{0.02cm} \begin{minipage}[c]{1.5cm} \centering 541,969\\0 / 24.31 \end{minipage} & \vspace{0.02cm} \begin{minipage}[c]{1.5cm} \centering 541,969\\0 / 1.77 \end{minipage} & \vspace{0.02cm} \begin{minipage}[c]{1.5cm} \centering 515,567\\0 / 122.72 \end{minipage} & \vspace{0.02cm} \begin{minipage}[c]{1.5cm} \centering 541,537\\5 / 7.98 \end{minipage} \\ 
    \midrule automatic-9.txt & \vspace{0.02cm} \begin{minipage}[c]{1.5cm} \centering 1,268,407\\4 / 46.25 \end{minipage} & \vspace{0.02cm} \begin{minipage}[c]{1.5cm} \centering 1,266,803\\0 / 123.23 \end{minipage} & \vspace{0.02cm} \begin{minipage}[c]{1.5cm} \centering 1,268,389\\5 / 4.63 \end{minipage} & \vspace{0.02cm} \begin{minipage}[c]{1.5cm} \centering 1,234,556\\0 / 123.09 \end{minipage} & \vspace{0.02cm} \begin{minipage}[c]{1.5cm} \centering 1,267,304\\0 / 123.04 \end{minipage} & \vspace{0.02cm} \begin{minipage}[c]{1.5cm} \centering 1,267,654\\17 / 104.16 \end{minipage} & \vspace{0.02cm} \begin{minipage}[c]{1.5cm} \centering 1,267,963\\0 / 2.12 \end{minipage} & \vspace{0.02cm} \begin{minipage}[c]{1.5cm} \centering 1,216,415\\0 / 123.11 \end{minipage} & \vspace{0.02cm} \begin{minipage}[c]{1.5cm} \centering 1,268,456\\0 / 4.44 \end{minipage} \\ 
    \midrule automatic-10.txt & \vspace{0.02cm} \begin{minipage}[c]{1.5cm} \centering 56,231\\0 / 122.58 \end{minipage} & \vspace{0.02cm} \begin{minipage}[c]{1.5cm} \centering 56,205\\0 / 122.29 \end{minipage} & \vspace{0.02cm} \begin{minipage}[c]{1.5cm} \centering 56,260\\0 / 1.60 \end{minipage} & \vspace{0.02cm} \begin{minipage}[c]{1.5cm} \centering 54,928\\0 / 122.30 \end{minipage} & \vspace{0.02cm} \begin{minipage}[c]{1.5cm} \centering 55,581\\0 / 122.27 \end{minipage} & \vspace{0.02cm} \begin{minipage}[c]{1.5cm} \centering 56,260\\0 / 5.59 \end{minipage} & \vspace{0.02cm} \begin{minipage}[c]{1.5cm} \centering 56,260\\0 / 1.31 \end{minipage} & \vspace{0.02cm} \begin{minipage}[c]{1.5cm} \centering 55,658\\0 / 122.28 \end{minipage} & \vspace{0.02cm} \begin{minipage}[c]{1.5cm} \centering 56,260\\0 / 4.78 \end{minipage} \\ 
   \bottomrule \end{tabular}

\caption{Local search results. For each instance, we show the number
  of crossings (first row), the iteration needed to reach the local
  optimum (second row, first value) and the run-time in seconds of our
  algorithm (second row, second value).  If no local optimum was found
  the number of iterations is -1. The best solution is highlighted in
  bold.}
\label{tab:results}
\end{table}
\end{landscape}



\end{document}
