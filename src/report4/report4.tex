\documentclass{scrartcl}
\usepackage{listings}
\usepackage{caption}
\usepackage{color}
\usepackage{booktabs}
\usepackage{lscape}
\usepackage{tabularx}

\newcounter{nalg}[section] % defines algorithm counter for section
\renewcommand{\thenalg}{\thesection .\arabic{nalg}} %defines appearance of the algorithm counter
\DeclareCaptionLabelFormat{algocaption}{Algorithm \thenalg} % defines a new caption label as Algorithm x.y

\lstnewenvironment{algorithm}[1][] %defines the algorithm listing environment
{   
    \refstepcounter{nalg} %increments algorithm number
    \captionsetup{labelformat=algocaption,labelsep=colon} %defines the
                                                          %caption
                                                          %setup for:
                                                          %it ises
                                                          %label
                                                          %format as
                                                          %the
                                                          %declared
                                                          %caption
                                                          %label above
                                                          %and makes
                                                          %label and
                                                          %caption
                                                          %text to be
                                                          %separated
                                                          %by a ':'
    \lstset{ %this is the stype
        frame=tB,
        numbers=left, 
        numberstyle=\tiny,
        basicstyle=\scriptsize, 
        keywordstyle=\color{black}\bfseries,
        escapeinside={(*}{*)},
        keywords={,input, output, return, datatype, function, in, if, else, foreach, while, begin, end, } 
        numbers=left,
        xleftmargin=.04\textwidth,
        #1 % this is to add specific settings to an usage of this environment (for instnce, the caption and referable label)
    }
}
{}





\author{Alexander Eisl (0250266), Peter Wiedermann (0025999)}

\date{\today}


\title{Heuristic Optimization Techniques \\ Exercise 3}

\begin{document}
\maketitle


\section{General Variable Neighborhood search}
\label{sec:general_vns}

In this programming exercise, we decided to search for optimal
solutions using an ant colony optimization (ACO) algorithm.


\subsection{Algorithm}

\begin{algorithm}[caption={Local search}]
    input: Graph 
    output: Improved solution
    begin
        initialize ant colony;
        for each (*$t \leftarrow 1,...,t_max$*) do
      	    for each ant (*$k = 1,...,m$*) do
               choose random start;
               walk through construction graph according to $p_{ij}$
            end for
        end for
        Apply pheromones;
        Evaporate pheromones;
    end
\end{algorithm}

Ants select the next vertex in the construction graph based on the
transition probabilities $p_{ij}$ from vertex $i$ to $j$. The
probabilities are calculated as a function of the pheromones
$\tau_{ij}$ and the local information $\nu_{ij}$.



%%% %%%%%%%%%%%%%%%%%%%%%%%%%%%%%%%%%%%%%%%%%%%%%%%%%%%%%
%%% Simple explanation of the construction graph
\subsection{Construction graph}
\label{construction_graph}
We follow the approach used in the previous exercises and build the
construction graph (CG) in a two-step procedure. The first part of our
CG represents the spine-order. The second part represents the edge
allocation. We do not store a representation of our construction graph
in memory, as this would be very large for the bigger problem
instances.

\begin{description}
  \item[spine-order] The CG in this step corresponds to a complete
    graph of the vertices. An ant can walk from each vertex to every other vertex, but
    visit each vertex only once. The order in which an ant visits the vertices
    corresponds to our spine order.

  \item[page-allocation] For each edge $j$, we create $k$ vertices
    which representation the page allocation of this edge. An ant can
    walk from the vertex of edge $j-1$ to $j$ by chosing $k$
    pathes. Depending on which path the ant choses, edge $j$ is on the
    corresponding page.
\end{description}

%%% %%%%%%%%%%%%%%%%%%%%%%%%%%%%%%%%%%%%%%%%%%%%%%%%%%%%%
%%% Simple explanation of the construction graph
\subsection{Pheromone model}
We implemented a pheromone model which works similarly for the two
parts of our construction graph. For each walk, we calculate the
number of crossings $c$ and increase the pheromones on each edge of the
construction graph by $1/c$.

%%% %%%%%%%%%%%%%%%%%%%%%%%%%%%%%%%%%%%%%%%%%%%%%%%%%%%%%
%%% Heuristic used for local information
\subsection{Heuristic for local information}
\label{local_information}

Local information is calculated different for the two parts of our construction graph.

\begin{description}

    \item{Spine-Order}: Local information is proportional to the number of edges of the
      nodes in the neighboorhood of $i$.              

    \item{Edge-Allocation}: Local information is proportional to the number of additional
      crossings an edge would introduce when added to page $k$. As this is path-dependent, local
      information needs to be evaluated for each step an ant takes in this part of the CG.
\end{description}

\subsection{Deamon Action}
\label{Deamon Action}
In addition to the regular pheromone updates, we implement a deamon
action. This can punish or reward 10 percent of the worst / best
edges. Our experiments indicate that only the version which gives
rewards yield better results.


%%% %%%%%%%%%%%%%%%%%%%%%%%%%%%%%%%%%%%%%%%%%%%%%%%%%%%%%5
%%% NEIGHBORHOODS HERE
\section{Experiments and Results}
We executed the code on a desktop computer with a Core i7 Quad-Core
CPU with 2.67Ghz and 24 GB of main memory. \\

We run a set of experiments to find out how the different parameters
($\alpha$ $\beta$) influence the results. We also evaluate how the
number runs and the size of our ant colony influence to results of our
optimization. \\


\begin{table}[!H]
\centering
\scriptsize  
% latex table generated in R 3.0.2 by xtable 1.7-4 package
% Sun Dec 13 09:20:29 2015
\begin{tabular}{rrrrrr}
  \hline
 & 5 & 10 & 20 & 50 & 100 \\ 
  \hline
100 & 24.50 & 20.00 & 15.20 & 11.10 & 11.10 \\ 
  500 & 18.30 & 16.50 & 13.90 & 12.20 & 10.30 \\ 
  1000 & 15.40 & 12.40 & 13.10 & 10.80 & 8.80 \\ 
  2000 & 14.30 & 11.50 & 11.40 & 9.40 & 8.80 \\ 
   \hline
\end{tabular}

\caption{This table shows the number of crossings for instance \texttt{automatic-4} for
different numbers of runs (columns) and ants (rows).}
\label{tab:runTable}
\end{table}


\begin{table}[!H]
\centering
\scriptsize  
% latex table generated in R 3.0.2 by xtable 1.7-4 package
% Sun Dec 13 15:44:06 2015
\begin{tabular}{rrrr}
  \hline
 & 0.1 & 1 & 2 \\ 
  \hline
0.1 & 13.50 & 8.90 & 5.10 \\ 
  1 & 14.20 & 8.80 & 4.90 \\ 
  2 & 9.50 & 7.00 & 3.60 \\ 
   \hline
\end{tabular}

\caption{This table shows the number of crossings for instance \texttt{automatic-4} for
different weights for $\alpha$ (rows) and $\beta$ (columns).}
\label{tab:alphabeta}
\end{table}


For computational reasons, we run most of our experiments on selected
(smaller) problem instances. Table~\ref{tab:runTable} shows the number
of crossings for different numbers of runs and ants. As expected, both
runs and ants improve the solution. However, especially in the
beginning an increase of runs seems to be more beneficial.\\

We tried different parameters for $\alpha$ and
$\beta$. Table~\ref{tab:alphabeta} shows the number of crossings for
instance \texttt{automatic-4} for $1000$ ants and $20$ runs.  Overall,
higher powers seem to favour good solutions. This seems intuitive as
it gives more weights to better local information / pheromones. For
this instance and parameterization (runs, ants), the parameter for
local information seems to be more important. \\

We also run a handful of experiments for the evaporation rate and find that
higher values tend to give better results.

Table~\ref{tab:results} shows the overall results. For the smaller
problem instances, we tend to find very good solutions compared to our
previous exercies. However, for the larger instances we are unable to
use a sufficient number of ants and runs. Given that the CGs of these
instances are very large we suspect that we would need a much larger
number of ants (and/or better local information) to find meaningful
solutions.

For selected instances, we executed our algorithm with an even larger
number of ants/runs and were able to achieve very promising
results. However, due to the run-time we did not run experiments and
cannot report averages and standard deviations of our results.


\begin{table}[!H]
\scriptsize
\centering
\begin{tabular}{cccccccc}
\toprule
 & & & \multicolumn{3}{c}{Crossings} & \multicolumn{2}{c}{Time} \\
Instance & Ants & Runs & avg & min & sd & avg & sd \\
\midrule
\texttt{auto-1} & 10,000 & 50 & 14.87 & 14 & 0.35 & 61.08 & 0.58  \\
\texttt{auto-2} & 10,000 & 50 & 0 & 1.13 & 0.83 & 54.00 & 0.59 \\
\texttt{auto-3} & 3,000  & 40 & 171 & 179.87 & 4.19 & 4.58 & 0.5 \\
\texttt{auto-4} & 10,000 & 40 & 2 & 4.00 & 1.19 & 72.89 & 1.94 \\
\texttt{auto-5} & 5,000 & 40 & 22 & 30.25 & 3.69 & 47.75 & 1.15 \\
\texttt{auto-6} & 10 & 2 & 8 Mio. & 8 Mio. & 1,810 & 111 & 2.46 \\
\texttt{auto-7} & 100 & 5 & 39,029 & 41,060 & 1,166 & 99.34 & 9.36 \\
\texttt{auto-8} & 100 & 5 & 1 Mio. & 1 Mio.  & 4,563 & 318.35 & 7.36 \\
\texttt{auto-9} & 20 & 5 & 1.4 Mio. & 1.47 Mio. & 34,775 & 149.38 & 4.19 \\
\texttt{auto-10} & 100 & 5 & 168,979 & 170,761 & 1,173.43 & 157.37 & 11.51 \\
\bottomrule

\end{tabular}
\caption{This table shows the crossing numbers and timings for all
  instances. We report the number of ants and the number of runs as
  these are vary across instances. We used $\alpha=2$, $\beta=2$ and a
  pheromone decay rate of $0.4$ for all runs.}
\label{tab:results}
\end{table}

\clearpage







\end{document}
