\documentclass{scrartcl}
\usepackage{listings}
\usepackage{caption}
\usepackage{color}
\usepackage{booktabs}
\usepackage{lscape}
\usepackage{tabularx}

\newcounter{nalg}[section] % defines algorithm counter for section
\renewcommand{\thenalg}{\thesection .\arabic{nalg}} %defines appearance of the algorithm counter
\DeclareCaptionLabelFormat{algocaption}{Algorithm \thenalg} % defines a new caption label as Algorithm x.y

\lstnewenvironment{algorithm}[1][] %defines the algorithm listing environment
{   
    \refstepcounter{nalg} %increments algorithm number
    \captionsetup{labelformat=algocaption,labelsep=colon} %defines the
                                                          %caption
                                                          %setup for:
                                                          %it ises
                                                          %label
                                                          %format as
                                                          %the
                                                          %declared
                                                          %caption
                                                          %label above
                                                          %and makes
                                                          %label and
                                                          %caption
                                                          %text to be
                                                          %separated
                                                          %by a ':'
    \lstset{ %this is the stype
        frame=tB,
        numbers=left, 
        numberstyle=\tiny,
        basicstyle=\scriptsize, 
        keywordstyle=\color{black}\bfseries,
        escapeinside={(*}{*)},
        keywords={,input, output, return, datatype, function, in, if, else, foreach, while, begin, end, } 
        numbers=left,
        xleftmargin=.04\textwidth,
        #1 % this is to add specific settings to an usage of this environment (for instnce, the caption and referable label)
    }
}
{}





\author{Alexander Eisl (0250266), Peter Wiedermann (0025999)}

\date{\today}


\title{Heuristic Optimization Techniques \\ Exercise 3}

\begin{document}
\maketitle


\section{General Variable Neighborhood search}
\label{sec:general_vns}

In this programming exercise, we implement ACO.


\subsection{Algorithm}

\begin{algorithm}[caption={Local search}]
    input: Graph 
    output: Improved solution
    begin
        initialize ant colony;
        for each (*$t \leftarrow 1,...,t_max$*) do
      	    for each ant (*$k = 1,...,m$*) do
               choose random start;
               walk through construction graph according to $p_{ij}$
            end for
        end for
        Apply pheromones;
        Evaporate pheromones;
    end
\end{algorithm}

Ants select the next vertex in the construction graph based on the
transition probabilities $p_{ij}$ from vertex $i$ to $j$. The
probabilities are calculated as a function of the pheromones
$\tau_{ij}$ and the local information $\nu_{ij}$.



%%% %%%%%%%%%%%%%%%%%%%%%%%%%%%%%%%%%%%%%%%%%%%%%%%%%%%%%5
%%% NEIGHBORHOODS HERE
\subsection{Construction graph}
\label{construction_graph}

We build our construction graph in a two-step procedure. The first
part of our construction graph represents the spine-order. The second
part represents the edge allocation.

[TODO: Illustration?]




\subsection{Local Information}
\label{local_information}

Local information is calculated different for the two parts of our construction graph.

\begin{description}

    \item{Spine-Order}: Local information is proportional to the number of edges of the
      nodes in the neighboorhood of $i$.
                
     \begin{equation*}
       \nu_{ij} = Number of edges of each vertex $j$
     \end{equation*}

     Alternatively, we could calculate a similarity index, e.g. (TODO could
     be more sophisticated)

     \begin{equation*}
       \nu_{ij} = \frac{Median of adjacent nodes of $i$}{Median of adjacent nodes of $j$}
     \end{equation*}
                

    \item{Edge-Allocation}: 

\end{description}


%%% %%%%%%%%%%%%%%%%%%%%%%%%%%%%%%%%%%%%%%%%%%%%%%%%%%%%%5
%%% NEIGHBORHOODS HERE
\section{Results}

We executed the code on a desktop computer with a Core i7 Quad-Core
CPU with 2.67Ghz and 24 GB of main memory. 

\paragraph{}
We set a timeout of 5 minutes for the calculations.  Because we
are relying on random calculations, we provide the summary statistics
of the results over 30 independent runs.


\paragraph{}
Table~\ref{tab:resultsOverall} shows the number of crossings for the
GVNS using the order of neighborhoods as described in Section~\ref{neighborhood_orders}.
The corresponding run-times are reported in Table~\ref{tab:Timing}.

We find that for the small instances, the GVNS finds better solutions
than the local search we used in the previous exercise. The timing in
Table~\ref{tab:Timing} shows that for larger instances (except for
instance 7) the maximum run-time is usually exhausted and, as a
result, we do not find large improvements compared to the local search
in the previous exercise.

\paragraph{Order of the neighborhoods}
We run two additional experiments where we change the order of the
neighborhoods.\footnote{see Section~\ref{neighborhood_orders} for the
  original order}

Tables~\ref{tab:resultsVns} and ~\ref{tab:TimingVns} show the results
when we reverse the order of the VNS neighborhoods used for the
shaking.\footnote{Unfortunately, we have to re-run our computations
  because we have overwritten our result files. Runtimes and crossing
  numbers were recorded separately and are correct. Because of the
  random nature of our algorithm, the crossing numbers of the
  solutions that we will submit on the cluster tomorrow might be
  different. Please note that the submission deadline on TUWEL seems
  to be too early (Sunday, 00:00) and different from the one mentioned
  in the assignment (Sunday, 23:55).}

As we can see, the reverse order of the VNS neighborhoods improves the
minimum number of crossings for many instances. If we are looking at
the timing results of instance 7, in Table~\ref{tab:TimingVns}, we see
that the variance in the runtime is higher than in the original order.
%%% removed 'significantly' as this could imply some statistical test which we didn't run
This might imply that we get a better minimum for the cost of
stability of the solution, originating from the fact that we start
exchanging 3 random vertices, than just 1 as in the original
order. However, the fact that we hit the maximum run-time in many
cases makes this difficult to interpret.


\begin{table}[!H]
  \centering
  \scriptsize
  % latex table generated in R 3.0.2 by xtable 1.7-4 package
% Sat Nov 21 12:12:19 2015
\begin{tabular}{lrrrrr}
  \toprule  & var & avg & sd & min & min\_run \\ 
  \midrule automatic-1.txt & 1.51 & 9.42 & 1.23 & 9 & 2 \\ 
  automatic-2.txt & 35.90 & 41.68 & 5.99 & 36 & 0 \\ 
  automatic-3.txt & 20.88 & 67.53 & 4.57 & 59 & 10 \\ 
  automatic-4.txt & 234.26 & 97.05 & 15.31 & 84 & 0 \\ 
  automatic-5.txt & 0.95 & 40.68 & 0.98 & 39 & 0 \\ 
  automatic-6.txt & 0.00 & 6,153,059.00 & 0.00 & 6,153,059 & 0 \\ 
  automatic-7.txt & 714,058.88 & 136,518.53 & 845.02 & 135,111 & 1 \\ 
  automatic-8.txt & 581,888.04 & 541,495.47 & 762.82 & 539,317 & 11 \\ 
  automatic-9.txt & 687,002.37 & 1,266,526.95 & 828.86 & 1,265,326 & 7 \\ 
  automatic-10.txt & 24,946.19 & 55,289.26 & 157.94 & 55,035 & 16 \\ 
   \bottomrule \end{tabular}

\caption{Number of crossings using default neighborhood order}
\label{tab:resultsOverall}
\end{table}

\begin{table}[!H]
  \centering
  \scriptsize
  % latex table generated in R 3.0.2 by xtable 1.7-4 package
% Sat Nov 21 16:37:50 2015
\begin{tabular}{lrrrr}
  \toprule  & var & avg & sd & min \\ 
  \midrule automatic-1.txt & 2.53 & 9.68 & 1.59 & 9 \\ 
  automatic-2.txt & 41.53 & 39.21 & 6.44 & 25 \\ 
  automatic-3.txt & 39.41 & 64.53 & 6.28 & 50 \\ 
  automatic-4.txt & 13.62 & 127.53 & 3.69 & 112 \\ 
  automatic-5.txt & 107.82 & 45.84 & 10.38 & 30 \\ 
  automatic-6.txt & 0.00 & 6,153,059.00 & 0.00 & 6,153,059 \\ 
  automatic-7.txt & 1,033,834.45 & 136,568.16 & 1,016.78 & 134,450 \\ 
  automatic-8.txt & 0.00 & 541,969.00 & 0.00 & 541,969 \\ 
  automatic-9.txt & 1,096,844.32 & 1,267,338.32 & 1,047.30 & 1,265,185 \\ 
  automatic-10.txt & 76,894.93 & 55,855.74 & 277.30 & 55,425 \\ 
   \bottomrule \end{tabular}

\caption{Number of crossings using reverse order of VNS neighborhood}
\label{tab:resultsVns}
\end{table}

\begin{table}[!H]
  \centering
  \scriptsize
  % latex table generated in R 3.0.2 by xtable 1.7-4 package
% Sat Nov 21 12:12:44 2015
\begin{tabular}{lrrrrr}
  \toprule  & var & avg & sd & min & min\_run \\ 
  \midrule automatic-1.txt & 2.35 & 16.47 & 1.53 & 12 & 0 \\ 
  automatic-2.txt & 19.15 & 37.89 & 4.38 & 36 & 0 \\ 
  automatic-3.txt & 27.64 & 64.79 & 5.26 & 56 & 6 \\ 
  automatic-4.txt & 116.13 & 137.37 & 10.78 & 108 & 6 \\ 
  automatic-5.txt & 30.85 & 47.32 & 5.55 & 40 & 0 \\ 
  automatic-6.txt & 954,027,528.83 & 6,011,850.26 & 30,887.34 & 5,962,740 & 16 \\ 
  automatic-7.txt & 52,922.03 & 137,904.84 & 230.05 & 137,553 & 4 \\ 
  automatic-8.txt & 5,544,075.63 & 504,782.95 & 2,354.59 & 501,545 & 12 \\ 
  automatic-9.txt & 13,573,379.20 & 1,188,287.47 & 3,684.21 & 1,185,819 & 4 \\ 
  automatic-10.txt & 28,478.53 & 55,428.00 & 168.76 & 55,178 & 18 \\ 
   \bottomrule \end{tabular}

\caption{Number of crossings using reverse order of VND neighborhood}
\label{tab:resultsVnd}
\end{table}

\clearpage

\begin{table}[H]
  \centering
  \scriptsize
  % latex table generated in R 3.0.2 by xtable 1.7-4 package
% Sat Nov 21 16:37:51 2015
\begin{tabular}{lrrrr}
  \toprule  & var & avg & sd & min \\ 
  \midrule automatic-1.txt & 0.00 & 0.03 & 0.00 & 0 \\ 
  automatic-2.txt & 0.00 & 0.02 & 0.01 & 0 \\ 
  automatic-3.txt & 0.05 & 0.61 & 0.22 & 0 \\ 
  automatic-4.txt & 0.00 & 0.08 & 0.01 & 0 \\ 
  automatic-5.txt & 0.00 & 0.06 & 0.01 & 0 \\ 
  automatic-6.txt & 0.26 & 309.22 & 0.51 & 309 \\ 
  automatic-7.txt & 712.56 & 277.61 & 26.69 & 228 \\ 
  automatic-8.txt & 0.01 & 302.82 & 0.09 & 303 \\ 
  automatic-9.txt & 0.00 & 303.16 & 0.06 & 303 \\ 
  automatic-10.txt & 0.00 & 302.35 & 0.04 & 302 \\ 
   \bottomrule \end{tabular}

\caption{Runtime in seconds using default neighborhood order}
\label{tab:Timing}
\end{table}

\begin{table}[H]
  \centering
  \scriptsize
  % latex table generated in R 3.0.2 by xtable 1.7-4 package
% Sat Nov 21 12:28:08 2015
\begin{tabular}{lrrrrr}
  \toprule  & var & avg & sd & min & min\_run \\ 
  \midrule automatic-1.txt & 0.00 & 0.02 & 0.00 & 0 & 0 \\ 
  automatic-2.txt & 0.00 & 0.02 & 0.01 & 0 & 13 \\ 
  automatic-3.txt & 0.22 & 0.74 & 0.47 & 0 & 2 \\ 
  automatic-4.txt & 0.00 & 0.05 & 0.02 & 0 & 8 \\ 
  automatic-5.txt & 0.00 & 0.12 & 0.04 & 0 & 5 \\ 
  automatic-6.txt & 0.01 & 309.07 & 0.11 & 309 & 12 \\ 
  automatic-7.txt & 894.86 & 275.49 & 29.91 & 216 & 6 \\ 
  automatic-8.txt & 0.00 & 302.81 & 0.03 & 303 & 0 \\ 
  automatic-9.txt & 0.06 & 303.23 & 0.24 & 303 & 6 \\ 
  automatic-10.txt & 0.00 & 302.34 & 0.01 & 302 & 6 \\ 
   \bottomrule \end{tabular}

\caption{Runtime in seconds using reverse order of VNS neighborhood}
\label{tab:TimingVns}
\end{table}

\begin{table}[H]
  \centering
  \scriptsize
  % latex table generated in R 3.0.2 by xtable 1.7-4 package
% Sat Nov 21 12:27:58 2015
\begin{tabular}{lrrrrr}
  \toprule  & var & avg & sd & min & min\_run \\ 
  \midrule automatic-1.txt & 0.00 & 0.01 & 0.00 & 0 & 2 \\ 
  automatic-2.txt & 0.00 & 0.04 & 0.01 & 0 & 16 \\ 
  automatic-3.txt & 0.12 & 0.74 & 0.35 & 0 & 13 \\ 
  automatic-4.txt & 0.00 & 0.06 & 0.02 & 0 & 7 \\ 
  automatic-5.txt & 0.00 & 0.08 & 0.01 & 0 & 4 \\ 
  automatic-6.txt & 0.90 & 309.33 & 0.95 & 309 & 16 \\ 
  automatic-7.txt & 0.00 & 301.15 & 0.01 & 301 & 0 \\ 
  automatic-8.txt & 0.00 & 302.81 & 0.02 & 303 & 6 \\ 
  automatic-9.txt & 0.06 & 303.22 & 0.24 & 303 & 16 \\ 
  automatic-10.txt & 0.00 & 302.34 & 0.02 & 302 & 18 \\ 
   \bottomrule \end{tabular}

\caption{Runtime in seconds using reverse order of VND neighborhood}
\label{tab:TimingVnd}
\end{table}







\end{document}
