\begin{table}
\centering
\begin{tabular}{llp{7cm}}
\toprule
\textbf{Variable} & \textbf{Type} & \textbf{Description} \\
\midrule
Pages & Array & Number of pages is constant; allows direct access to each element \\
Page & List of edges & dynamic; only sequential access needed; each time we add an edge we incrementally update the number of crossings \\
Spine Order & Array & does not change often in our implementation; easy access. To be able to calculate
the number of crossings efficiently we use an additional spine order map.\\

\bottomrule 

\end{tabular}

\caption{This table shows structure of our solution.}
\label{tab:sol}
\end{table}


%% hier noch ein paar notes bzgl des report:




%%     solution representation

%%         in pseudo strukturen angegeben, analog zu pseudocode

%%         pages : array // weil anzahl konstant und so schneller zugriff auf alle arten

%%         page : edges als liste: weil dynamisch und immer nur sequentieller zugriff
%%                crossings werden abgespeichert damit nicht immer wieder berechnet,
%%                also inkrementel mit allokation
              
%%         spine_order : array weil selten veraendert und so beliiebeger yugriff

%%         spine_order_map: schneller zugriff fuer die ordnungszahl eines vertex um
%%                          die crossing berechnung effizient zu machen
