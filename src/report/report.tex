\documentclass{scrartcl}
\usepackage{listings}
\usepackage{caption}
\usepackage{color}
\usepackage{booktabs}

\newcounter{nalg}[section] % defines algorithm counter for section
\renewcommand{\thenalg}{\thesection .\arabic{nalg}} %defines appearance of the algorithm counter
\DeclareCaptionLabelFormat{algocaption}{Algorithm \thenalg} % defines a new caption label as Algorithm x.y

\lstnewenvironment{algorithm}[1][] %defines the algorithm listing environment
{   
    \refstepcounter{nalg} %increments algorithm number
    \captionsetup{labelformat=algocaption,labelsep=colon} %defines the
                                                          %caption
                                                          %setup for:
                                                          %it ises
                                                          %label
                                                          %format as
                                                          %the
                                                          %declared
                                                          %caption
                                                          %label above
                                                          %and makes
                                                          %label and
                                                          %caption
                                                          %text to be
                                                          %separated
                                                          %by a ':'
    \lstset{ %this is the stype
        frame=tB,
        numbers=left, 
        numberstyle=\tiny,
        basicstyle=\scriptsize, 
        keywordstyle=\color{black}\bfseries,
        escapeinside={(*}{*)},
        keywords={,input, output, return, datatype, function, in, if, else, foreach, while, begin, end, } 
        numbers=left,
        xleftmargin=.04\textwidth,
        #1 % this is to add specific settings to an usage of this environment (for instnce, the caption and referable label)
    }
}
{}





\author{Alexander Eisl (0250266), Peter Wiedermann (0025999)}

\date{\today}


\title{Heuristic Optimization Techniques \\ Exercise 1}

\begin{document}
\maketitle


\section{Deterministic Construction Heuristic}
\label{sec:deterministic}
In the first part of this programming exercise, we develop a
deterministic construction heuristic to solve the K-page crossing
number minimization problem. The following listing shows the
pseudo-code of our algorithm.

\begin{algorithm}[caption={Deterministic construction heuristic}]
    input: Graph with given spine-order and set of edges
    output: Graph with new spine-order and edge partition

    begin
       start with empty solution
       spine-order (*$\leftarrow$*) sort(spine order)
       while solution not complete
           add edge to page that creates the fewest crossings
       end 
    end
\end{algorithm}

Our heuristic construct the solution in two parts:

\begin{itemize}
\item In the first part, we create a new spine-order by sorting the
  vertices according to the number of neighbors of each vertex. The
  idea of this approach is to get vertices with many neighbors closer
  together to minimize the number of crossings the edges of these
  vertices can potentially produce.

\item In the second part of the heuristic, we add edges to the pages
  of our solution using a greedy approach. Edges are added in the
  order of the adjacency matrix (arbitrary).  Each edge is added to
  the first page where it does not introduce additional crossings.  If
  we cannot find such a page, we add the edge to the page where we
  introduce the minimum number of crossings.
\end{itemize}

%%% Argue why we generate the spine-order first and not sequentially
We decided to generate the spine-order in the beginning and then
decide on the edge-partinioning independently. Separating the two
problems makes the algorithm simpler and computationally faster.

% It also allows for a simple way to incorporate randomness 
% in our construction heuristic.
We will mix spine-order and edge-page allocation during our randomized
variant desribed in Section~\ref{sec:randomization}.



\section{Randomization}
\label{sec:randomization}
In this section, we explain how we extend our approach to a
randomized/multi-solution heuristic. We do this by randomly
permutating a subset of the vertices on the spine. Thus, the result of
our spine-sorting algorithm is subject to random variations.

We first start with a spine-porder as described in our deterministic
variant and apply the following randomized variant:

\begin{itemize}
\item We randomly permutate $n$ vertices $k$ times, to create $k$
  different initial spine-orders. For each of these $k$ spine-orders
  we can run the same edge-allocation algorithm as described in
  section \ref{sec:deterministic}.

\item We then compute the number of
crossing and -- from the list of generated solutions -- select the one
resulting in the lowest number of edge-crossings. 
\end{itemize}


%%% Why this is not random search
%Our algorithm also does not degenerate into random search, as we limit
%the number of random permutations that are allowed.

Our algorithm does not degenerate into random search because we try to
improve our deterministic approach.


\section{Solution Representation}

\begin{table}
\centering
\begin{tabular}{llp{7cm}}
\toprule
\textbf{Variable} & \textbf{Type} & \textbf{Description} \\
\midrule
Pages & Array & Number of pages is constant; allows direct access to each element \\
Page & List of edges & dynamic; only sequential access needed; each time we add an edge we incrementally update the number of crossings \\
Spine Order & Array & does not change often in our implementation; easy access. To be able to calculate
the number of crossings efficiently we use an additional spine order map.\\

\bottomrule 

\end{tabular}

\caption{This table shows structure of our solution.}
\end{table}


%% hier noch ein paar notes bzgl des report:




%%     solution representation

%%         in pseudo strukturen angegeben, analog zu pseudocode

%%         pages : array // weil anzahl konstant und so schneller zugriff auf alle arten

%%         page : edges als liste: weil dynamisch und immer nur sequentieller zugriff
%%                crossings werden abgespeichert damit nicht immer wieder berechnet,
%%                also inkrementel mit allokation
              
%%         spine_order : array weil selten veraendert und so beliiebeger yugriff

%%         spine_order_map: schneller zugriff fuer die ordnungszahl eines vertex um
%%                          die crossing berechnung effizient zu machen
 

Table~\ref{tab:sol} shows the basic
structure of our solution representation.

\section{Results}
%%% den doppelten Verweis auf Table 3 gestrichen
We executed the code on a desktop computer with a Core i7 Quad-Core
CPU with 2.67Ghz and 24 GB of main memory. Table~\ref{tab:results}
shows the results of our randomized algorithm, where we have choosen
to perform $n=4$ permutations of vertices to generete $k=5$ different
solutions.

The solutions we generate using the random approach are slightly
better than the solutions generated by our deterministic routine. This
is not surpring, as we can make sure that the randomized heuristic
produces at least the same result as the deterministic algorithm by
including the deterministic spine-order in our set of solution
candidates.


\begin{table}[h]

\centering
% latex table generated in R 2.15.1 by xtable 1.5-6 package
% Sun Oct 25 11:09:42 2015
\begin{tabular}{l|rrr|rrr|}
  \toprule   & \multicolumn{2}{c}{Objective} & \multicolumn{2}{c}{Runtime} \\ & min & avg & sd & min & avg & sd \\ 
  \midrule automatic-1.txt & 17.00 & 17.24 & 0.43 & 0.00 & 0.00 & 0.00 \\ 
  automatic-2.txt & 48.00 & 48.00 & 0.00 & 0.00 & 0.00 & 0.00 \\ 
  automatic-3.txt & 152.00 & 152.00 & 0.00 & 0.00 & 0.00 & 0.00 \\ 
  automatic-4.txt & 169.00 & 169.00 & 0.00 & 0.00 & 0.00 & 0.00 \\ 
  automatic-5.txt & 74.00 & 74.00 & 0.00 & 0.00 & 0.00 & 0.00 \\ 
   \bottomrule \end{tabular}

\caption{This table shows the results of our randomized construction
  heuristic. The first three columns show the minimum, mean and
  standard deviation of the number of crossings, the last three
  columns show the minimum, mean and standard deviation of the runtime
  in seconds.}
\caption{This table shows the results of our randomized construction heuristic.}

\label{tab:results}
\end{table}


\end{document}
