\documentclass{scrartcl}
\usepackage{listings}
\usepackage{caption}
\usepackage{color}
\usepackage{booktabs}
\usepackage{lscape}
\usepackage{tabularx}

\newcounter{nalg}[section] % defines algorithm counter for section
\renewcommand{\thenalg}{\thesection .\arabic{nalg}} %defines appearance of the algorithm counter
\DeclareCaptionLabelFormat{algocaption}{Algorithm \thenalg} % defines a new caption label as Algorithm x.y

\lstnewenvironment{algorithm}[1][] %defines the algorithm listing environment
{   
    \refstepcounter{nalg} %increments algorithm number
    \captionsetup{labelformat=algocaption,labelsep=colon} %defines the
                                                          %caption
                                                          %setup for:
                                                          %it ises
                                                          %label
                                                          %format as
                                                          %the
                                                          %declared
                                                          %caption
                                                          %label above
                                                          %and makes
                                                          %label and
                                                          %caption
                                                          %text to be
                                                          %separated
                                                          %by a ':'
    \lstset{ %this is the stype
        frame=tB,
        numbers=left, 
        numberstyle=\tiny,
        basicstyle=\scriptsize, 
        keywordstyle=\color{black}\bfseries,
        escapeinside={(*}{*)},
        keywords={,input, output, return, datatype, function, in, if, else, foreach, while, begin, end, } 
        numbers=left,
        xleftmargin=.04\textwidth,
        #1 % this is to add specific settings to an usage of this environment (for instnce, the caption and referable label)
    }
}
{}





\author{Alexander Eisl (0250266), Peter Wiedermann (0025999)}

\date{\today}


\title{Heuristic Optimization Techniques \\ Exercise 5 and 6}

\begin{document}
\maketitle


\section{Hybrid Methods - ACO + Local Search}
\label{sec:hybdrin_methods}

In this programming exercise, we decided to search for optimal
solutions using our ant colony optimization (ACO) from the last assignment
in combination with Variable Neighborhood Search.

\subsection{Algorithm and Parametrization}

The algorithm is similar to the variant in the last assignment, but in
addition we are optimizing the solutions of
the ants after each run before we apply the pheromones:

\begin{algorithm}[caption={Ant Colony - VNS Hybrid}]
    input: Graph 
    output: Improved solution
    begin
        initialize ant colony;
        for each (*$t \leftarrow 1,...,t_{max}$*) do
      	    for each ant (*$k = 1,...,m$*) do
               choose random start;
               walk through construction graph according to (*$p_{ij}$*)
            end for

            Apply Variable Neighborhood Descent for all ants;
            Apply Generalized Variable Neighborhood Search for all ants;
            Apply pheromones;
            Evaporate pheromones;

        end for
    end
\end{algorithm}

Note that we execute a Variable Neighborhood Descent before we apply the 
Generalized Variable Neigborhood Search, which itself executes the 
same VND after a stochastic move.
The reason for this approach is that each ant solution is already generated by a stochastic algorithm 
and we first want to improve this randomized results deterministically before we apply addittional 
shaking.

\subsubsection{Ant Colony Parametrization}
\label{ant_colony}

For basic parameters we are using the configuration we have chosen for Assignment 4:
\begin{itemize}
\item $\alpha: 2 $
\item $\beta: 2$
\item evaporation rate: 0.4
\item $t_{max}=3$
\end{itemize}
In addition we changed the random distribution for transition probabilities used by the ants for their 
way through the construction graph
by putting a higher weight/probability on local better solutions. 
The reason for that is, 
that we already know from our experiments for the last assignment that we have not enough calculation
power for a lot of runs (iterations of t), especially if we take the additional time for the
variable neighborhood search into account.

\subsubsection{Variable Neigborhood Search}

We are using the Generalized Variable Neighborhood Search as developed in Assignment~3.
As described in the algorithm section, we use the Variable Neighborhood Descent in the Generalized
Search as well as directly on the ant solutions.

For the \textbf{stochastic moves} we use the following neighborhoods:

\begin{description}
\item[1-node-move]: Defined as all subsets where 1 vertex of the initial solution is moved to another position.
\item[2-node-move] 
Defined as all subsets where 2 vertices of the initial solution are moved to other positions.
\end{description}
 
For the \textbf{Variable Neighborhood Descent} we are using the NEXT step function, as well as the following 
neighborhoods:

\begin{description}
\item[1-edge move] This neighborhood consists of all solutions where one edge is moved to a different page. 
\item[1-node edge move] This neighborhood consists of all solutions where the edges of
	one vertex on a specific page are moved to all different pages. 
\end{description}



\subsection{Experiments and Results}
We executed the code on a desktop computer with a Core i7 Quad-Core
CPU with 2.67Ghz and 24 GB of main memory. \\


We have set a timeout to 5 minutes for the variable neighborhood
search, where the time measurement includes the calculation time for
the ant colony part as well.  The time-based stopping criteria is only
executed in the neighborhood search parts of the heuristic.\\

Table \ref{tab:results} shows the results of the Ant-Colony-GVNS
Hybrid as well as the results without Generalized Variable
Neighborhood search.  As expected the Neighborhood Search extension is
an improvement to our heuristic, at least for many of the smaller instances. \\

An interesting fact is that the variant without GVNS is even better
performing than the heuristic of the last assignment. The reason for
that is that we changed the random distribution for the ant path
finding as described in section \ref{ant_colony} which is better
performing under the limited circumstances that we have to execute our
heuristic. The real power of the ant-colony could unfold in a
parallelized environment, where we could increase the number of runs
($t_{max}$) and ants.  If we would further increase the probability
for local better solutions we would degenerate to a randomized greedy
construction heuristic. \\

So as an additional insight of this task we can say that a randomized
greedy construction heuristic with a subsequent variable neighborhood
search seems to be a very promising approach for the k-page crossing
number minimization problem.



\begin{table}
\scriptsize
\begin{tabular}{lccccc}
  \toprule  & min & avg & sd & avgTime & sdTime \\ 
  \midrule automatic-1.txt & 9 / 12 & 9.38 / 12.63 & 0.74 / 0.52 & 432.29 / 76.89 & 2.09 / 1.03 \\ 
  automatic-2.txt & 0 / 2 & 0.25 / 5.75  & 0.46 / 2.38 & 438.38 / 66.26 & 1.22 / 0.99 \\ 
  automatic-3.txt & 63 / 80 & 68.875 / 87.13 & 2.80 / 3.52 & 916.39 / 463.49 & 6.98 / 6.39 \\ 
  automatic-4.txt & 5 / 3 & 6.875 / 6.625 & 1.13 / 2.615 & 444.67 / 86.45 & 2.17 / 1.91 \\ 
  automatic-5.txt & 10 / 19 & 11.75 / 23.38 & 1.58 / 2.13 & 394.09 / 56.94 & 1.57 / 0.87 \\ 
  automatic-6.txt & 7,388,672 / 7,442,471 & 7,425,825 / 7,448,006 & 20,228 / 3,185 & 387.24 / 124.00 & 1.72 / 1.16 \\ 
  automatic-7.txt & 40,953 / 34,222 & 44,370 / 38,430 & 2,092 / 2,185 & 445.33 / 127.77 & 2.28 / 1.08 \\ 
  automatic-8.txt & 1,008,639 / 1,007,304 & 1,014,500 / 1,015,000  & 3,317 / 4,124 & 634.86 / 410.49 & 4.80 / 1.58 \\ 
  automatic-9.txt & 1,177,866 / 1,125,711 & 1,204,371 / 1,197,395 & 14,501 / 40,997 & 449.14 / 180.57 & 8.69 / 6.35 \\ 
  automatic-10.txt & 150,320 / 152,032 & 152,916 / 153,774 & 1,432 / 1,104 & 819.98 / 433 & 3.19 / 1.40 \\ 
  \bottomrule 
\end{tabular}
\caption{Crossings: GVNS / without GVNS}
\label{tab:results}
\end{table}

\subsection{Parameter tuning}
In exercise, we tuned the parameter of the ACO used in the previous assignment. In order to be able to validate
the tuned parameters out-of-sample we generated a set of 40 random tuning instances. In order to be able to use
a reasonable number of ants we decided to simulate reasonably small graphs. 

\begin{itemize}
  \item Number of vertices: between 10 and 15 Vertices
  \item Number of edges: between 25 and 40 edges
  \item Number of pages: between 3 and 5 pages
\end{itemize}

We decided not to tune the number of ants and the number of runs as we
expect the results to be monotone in this parameter. Also, we would
not be able to use an ``optimal'' number of ants/runs that was
determined on a small instance on bigger instances. Thus, we used the following tuning parameters:

\begin{itemize}
  \item alpha: Real value between 0 and 3
  \item beta: Real value between 0 and 3
  \item use-daemon: Either 0 or 1
  \item evaporation rate: Real value between 0 and 1
\end{itemize}

\end{document}
