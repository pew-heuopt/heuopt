\documentclass{scrartcl}
\usepackage{listings}
\usepackage{caption}
\usepackage{color}
\usepackage{booktabs}
\usepackage{lscape}
\usepackage{tabularx}

\newcounter{nalg}[section] % defines algorithm counter for section
\renewcommand{\thenalg}{\thesection .\arabic{nalg}} %defines appearance of the algorithm counter
\DeclareCaptionLabelFormat{algocaption}{Algorithm \thenalg} % defines a new caption label as Algorithm x.y

\lstnewenvironment{algorithm}[1][] %defines the algorithm listing environment
{   
    \refstepcounter{nalg} %increments algorithm number
    \captionsetup{labelformat=algocaption,labelsep=colon} %defines the
                                                          %caption
                                                          %setup for:
                                                          %it ises
                                                          %label
                                                          %format as
                                                          %the
                                                          %declared
                                                          %caption
                                                          %label above
                                                          %and makes
                                                          %label and
                                                          %caption
                                                          %text to be
                                                          %separated
                                                          %by a ':'
    \lstset{ %this is the stype
        frame=tB,
        numbers=left, 
        numberstyle=\tiny,
        basicstyle=\scriptsize, 
        keywordstyle=\color{black}\bfseries,
        escapeinside={(*}{*)},
        keywords={,input, output, return, datatype, function, in, if, else, foreach, while, begin, end, } 
        numbers=left,
        xleftmargin=.04\textwidth,
        #1 % this is to add specific settings to an usage of this environment (for instnce, the caption and referable label)
    }
}
{}





\author{Alexander Eisl (0250266), Peter Wiedermann (0025999)}

\date{\today}


\title{Heuristic Optimization Techniques \\ Exercise 5 and 6}

\begin{document}
\maketitle


\section{Hybrid Methods - ACO + Local Search}
\label{sec:hybdrin_methods}

In this programming exercise, we decided to search for optimal
solutions using our ant colony optimization (ACO) from the last assignment
in combination with Variable Neighborhood Search.

\subsection{Algorithm and Parametrization}

The algorithm is similar to the variant in the last assignment, but in
addition we are optimizing the solutions of
the ants after each run before we apply the pheromones:

\begin{algorithm}[caption={Ant Colony - VNS Hybrid}]
    input: Graph 
    output: Improved solution
    begin
        initialize ant colony;
        for each (*$t \leftarrow 1,...,t_{max}$*) do
      	    for each ant (*$k = 1,...,m$*) do
               choose random start;
               walk through construction graph according to (*$p_{ij}$*)
            end for
        end for
        Apply Variable Neighborhood Search for all ants;
        Apply pheromones;
        Evaporate pheromones;
    end
\end{algorithm}


\subsubsection{Ant Colony Parametrization}
\label{ant_colony}

We are using the configuration we have chosen for Assignment 4:
\begin{itemize}
\item $\alpha: 2 $
\item $\beta: 2$
\item evaporation rate: 0.4
\end{itemize}
\subsubsection{Variable Neigborhood Search}

We are using the Generalized Variable Neighborhood Search as developed in assignment 3.
We have set a timeout to 5 minutes for the search.

For the stochastic moves we use the following neigborhoods:

\begin{description}
\item[1-node-move]: Defined as all subsets where 1 vertex of the initial solution is moved to another position.
\item[2-node-move] 
Defined as all subsets where 2 vertices of the initial solution are moved to other positions.
\end{description}
 
For the Variable Neighborhood decent we are using the NEXT best step function, as well as the following 
neighborhoods:

\begin{description}
\item[1-edge move] This neighborhood consists of all solutions where one edge is moved to a different page. 
\item[1-node edge move] This neighborhood consists of all solutions where the edges of
	one vertex on a specific page are moved to all different pages. 
\end{description}



\subsection{Experiments and Results}
We executed the code on a desktop computer with a Core i7 Quad-Core
CPU with 2.67Ghz and 24 GB of main memory. \\

We run a set of experiments to find out how the different parameters
($\alpha$ $\beta$) influence the results. We also evaluate how the
number runs and the size of our ant colony influence to results of our
optimization. \\


\end{document}
